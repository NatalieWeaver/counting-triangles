\documentclass[tikz]{standalone}
\usepackage{tikz}
\usetikzlibrary{calc,intersections,through,backgrounds}
\usepackage{xparse}

\begin{document}

% SET VALUES OF n AND m HERE!!
\newcommand{\n}{3}
\newcommand{\m}{3}

% Make a command for drawing triangle. This depends on n and m.
\newcommand{\triangleoutline}{
    % Draw a frame around the picture so we don't run up on the edges of the pdf document
    \draw (-2, -2) -- (12, -2) -- (12, 12) -- (-2, 12) -- cycle;
    
    % Count will go here later:
    %\node[align = left] at (7, 9) {How many triangles are there?};
    
    % A is the top vertex with n + 1 lines coming out of it
    \coordinate (A) at (0, 10);
    \node[label = 90: A] at (A) {};
    \foreach \z in {1,...,\n}
     {
           \coordinate ({\z}c{0}) at (A);
           %\node[label = 45: {\z}c{0}] at ({\z}c{0}) {};
           \node[] at ({\z}c{0}) {};
     }

    % B is the right-side vertex with m + 1 lines coming out of it
    \coordinate (B) at (10, 0);
    \node[label = 0: B] at (B) {};
    \foreach \z in {1,...,\m}
     {
           \coordinate ({0}c{\z}) at (B);
           %\node[label = 45: {0}c{\z}] at ({0}c{\z}) {};
           \node[] at ({0}c{\z}) {};
     }
    % Draw a line from A to B and name this line A--B
    \draw [name path = AB] (A) -- (B);
    
    % Draw and name the additional n lines coming out of A
    \foreach \a in {1, 2, ..., \n}
    {
    	\coordinate (a\a) at ({10-(10/\n)*\a}, 0);
    	%\node[label = -90: a\a] at (a\a) {};
	\node[] at (a\a) {};
    	\draw [name path global/.expanded=A\a] (A) -- (a\a);
    }
    
    % Draw and name the additional m lines coming out of B
    \foreach \b in {1, 2, ..., \m}
    {
    	\coordinate (b\b) at (0, {10-(10/\m)*\b});
    	%\node[label = 180: b\b] at (b\b) {};
	\node[] at (b\b) {};
    	\draw [name path global/.expanded=B\b] (B) -- (b\b);
    }
    
}

\newcommand{\nameintersectionpoints}{
	% Find and name the intersection points. Each of these points C_i corresponds to a triangle ABC_i
         \foreach \a in {1,...,\n}
          {
          	\foreach \b in {1,...,\m}
                 {                     
                     	\path [name intersections={of={A\a} and {B\b}, by = {\a}C{\b}}];
                     	\coordinate ({\a}c{\b}) at ({\a}C{\b});
                     	%\node[label = 45: {\a}c{\b}] at ({\a}c{\b}) {};
			\node[] at ({\a}c{\b}) {};                   
                  }            
          }
}

\begin{tikzpicture}
	\draw (-2, -2) -- (12, -2) -- (12, 12) -- (-2, 12) -- cycle;
	\draw (0,0) -- (10, 0) -- (0, 10) -- cycle;
	\node[label = 90: A] at (0, 10) {};
	\node[label = 0: B] at (10, 0) {};
\end{tikzpicture}

\renewcommand{\n}{2}
\renewcommand{\m}{2}
\begin{tikzpicture}
	\triangleoutline
\end{tikzpicture}

\renewcommand{\n}{3}
\renewcommand{\m}{3}
\begin{tikzpicture}
	\triangleoutline
\end{tikzpicture}

\renewcommand{\n}{4}
\renewcommand{\m}{4}
\begin{tikzpicture}
	\triangleoutline
\end{tikzpicture}

\renewcommand{\n}{5}
\renewcommand{\m}{5}
\begin{tikzpicture}
	\triangleoutline
\end{tikzpicture}

\renewcommand{\n}{6}
\renewcommand{\m}{6}
\begin{tikzpicture}
	\triangleoutline
\end{tikzpicture}

\end{document}
